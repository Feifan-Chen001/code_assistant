% !TeX program = xelatex
% !TeX encoding = UTF-8
\documentclass[11pt,a4paper]{article}
\usepackage{array}
\newcolumntype{L}[1]{>{\raggedright\arraybackslash}p{#1}}
\usepackage{xurl}
\usepackage{longtable}
\usepackage{booktabs}
\newcolumntype{T}[1]{>{\ttfamily\raggedright\arraybackslash}p{#1}}
\newcommand{\codepath}[1]{\path{#1}}

% ===== 中文与字体(支持UTF-8和中文路径) =====
\usepackage{xeCJK}
\setCJKmainfont{SimSun}
\setmonofont{Consolas}
\setCJKmonofont{SimSun}
\usepackage{fontspec}

% ===== 页面与排版 =====
\usepackage[a4paper,margin=2.2cm]{geometry}
\usepackage{setspace}
\setstretch{1.12}
\usepackage{microtype}

% ===== 数学/符号 =====
\usepackage{amsmath,amssymb}

% ===== 链接 =====
\usepackage[colorlinks=true,linkcolor=blue,urlcolor=blue,citecolor=blue]{hyperref}

% ===== 表格 =====
\usepackage{tabularx}
\usepackage{multirow}
\newcolumntype{Y}{>{\raggedright\arraybackslash}X}
\newcolumntype{C}{>{\centering\arraybackslash}X}

% ===== 代码块 =====
\usepackage{xcolor}
\usepackage{listings}
\lstset{
  basicstyle=\ttfamily\small,
  columns=fullflexible,
  breaklines=true,
  breakatwhitespace=true,
  showstringspaces=false,
  frame=single,
  framerule=0.3pt,
  rulecolor=\color{black!25},
  xleftmargin=1.2em,
  xrightmargin=0.6em,
  aboveskip=0.8em,
  belowskip=0.8em
}

% ===== 标题格式 =====
\usepackage{titlesec}
\titleformat{\section}{\Large\bfseries}{\thesection}{0.6em}{}
\titleformat{\subsection}{\large\bfseries}{\thesubsection}{0.6em}{}
\titleformat{\subsubsection}{\normalsize\bfseries}{\thesubsubsection}{0.6em}{}

\begin{document}

\begin{center}
{\LARGE \textbf{CodeAssistant 报告}}\\[4pt]
{\normalsize (由自动审查与测试生成系统输出)}
\end{center}

\vspace{0.5em}
\tableofcontents
\vspace{0.8em}
\hrule
\vspace{1.2em}

\section{代码审查(Review)}

\subsection{概览}
\begin{itemize}
  \item 问题总数:0
  \item 高/中/低:0 / 0 / 0
  \item 工具数:0
  \item DS 规则命中总数:0
\end{itemize}

\subsection{复杂度摘要}
\begin{itemize}
  \item 来源:Radon Cyclomatic Complexity(CC)
  \item 说明:等级通常为 A(简单)到 F(复杂),括号内为复杂度分数
\end{itemize}

\small
\setlength{\emergencystretch}{2em}
\begin{longtable}{T{6.6cm} L{1.1cm} T{5.2cm} L{0.9cm} r}
\toprule
文件 & 类型 & 符号(函数/方法) & 等级 & 分数 \\
\midrule
\endfirsthead
\toprule
文件 & 类型 & 符号(函数/方法) & 等级 & 分数 \\
\midrule
\endhead
\midrule
\multicolumn{5}{r}{(续下页)}\\
\endfoot
\bottomrule
\endlastfoot
\codepath{D:/code\_assistant/Git\_repo/zedr\_\_clean-code-python/conftest.py} & M & ReadmeItem.runtest & B & 6 \\
\codepath{D:/code\_assistant/Git\_repo/zedr\_\_clean-code-python/conftest.py} & C & ReadmeFile & A & 3 \\
\codepath{D:/code\_assistant/Git\_repo/zedr\_\_clean-code-python/conftest.py} & C & ReadmeItem & A & 3 \\
\codepath{D:/code\_assistant/Git\_repo/zedr\_\_clean-code-python/conftest.py} & F & pytest\_collect\_file & A & 2 \\
\codepath{D:/code\_assistant/Git\_repo/zedr\_\_clean-code-python/conftest.py} & M & ReadmeFile.collect & A & 2 \\
\codepath{D:/code\_assistant/Git\_repo/zedr\_\_clean-code-python/conftest.py} & F & fake\_print & A & 1 \\
\codepath{D:/code\_assistant/Git\_repo/zedr\_\_clean-code-python/conftest.py} & F & \_with\_patched\_sleep & A & 1 \\
\codepath{D:/code\_assistant/Git\_repo/zedr\_\_clean-code-python/conftest.py} & C & MyPyValidationError & A & 1 \\
\codepath{D:/code\_assistant/Git\_repo/zedr\_\_clean-code-python/conftest.py} & M & ReadmeItem.\_\_init\_\_ & A & 1 \\
\codepath{D:/code\_assistant/Git\_repo/zedr\_\_clean-code-python/conftest.py} & M & ReadmeItem.repr\_failure & A & 1 \\
\codepath{D:/code\_assistant/Git\_repo/zedr\_\_clean-code-python/conftest.py} & M & ReadmeItem.reportinfo & A & 1 \\
\end{longtable}
\normalsize

\section{测试生成(TestGen)}

\subsection{指标}
\begin{itemize}
  \item 写入测试文件数:0
  \item 覆盖函数数:0
  \item 输出目录:\codepath{D:/code\_assistant/Git\_repo/zedr\_\_clean-code-python/reports/zedr\_\_clean-code-python/generated\_tests}
\end{itemize}

\subsection{覆盖率报告}
\small
\setlength{\emergencystretch}{2em}
\begin{longtable}{T{7.2cm} r r r T{5.2cm}}
\toprule
Name & Stmts & Miss & Cover & Missing \\
\midrule
\endfirsthead
\toprule
Name & Stmts & Miss & Cover & Missing \\
\midrule
\endhead
\midrule
\multicolumn{5}{r}{(续下页)}\\
\endfoot
\bottomrule
\endlastfoot
\codepath{conftest.py} & 46 & 26 & 43\% & 19, 24-25, 33-35, 42-47, 60-61, 65-78, 82, 89 \\
\textbf{TOTAL} & 46 & 26 & 43\% &  \\
\end{longtable}
\normalsize

\end{document}